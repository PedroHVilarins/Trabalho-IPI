\documentclass[12pt,a4paper]{article}
\usepackage[utf8]{inputenc}
\usepackage[T1]{fontenc}
\usepackage[portuguese]{babel}
\usepackage{amsmath,amssymb}
\usepackage{graphicx}
\usepackage{float}
\usepackage{booktabs}
\usepackage{geometry}
\usepackage{caption}
\usepackage{subcaption}
\usepackage{hyperref}
\usepackage{algorithm}
\usepackage{algpseudocode}

\geometry{margin=2.0cm}

\title{\textbf{Análise Morfológica de Imagens para Detecção de Bordas Endocárdicas do Ventrículo Esquerdo em Ecocardiogramas 2D}}
\author{Gustavo de Sousa Santos - 222011534 \\ Jose Eduardo Lindoso - 160152658 \\ Pedro Henrique Da Costa Vilarins  - 180114441\\
\small Universidade de Brasília (UnB)}
\date{Dezembro de 2025}

\begin{document}

\maketitle

\begin{abstract}
Este relatório apresenta a implementação e avaliação do método semi-automático para detecção de bordas endocárdicas em imagens ecocardiográficas proposto por Choy e Jin (1996). O método combina técnicas de análise morfológica de imagens com o operador Laplaciano da Gaussiana (LoG) para redução de ruído e detecção de bordas. A implementação foi validada utilizando o dataset CAMUS, obtendo um coeficiente Dice médio de 0.798 $\pm$ 0.066 e erro RMS de 15.74 $\pm$ 1.54 pixels. Os resultados demonstram a eficácia do método clássico de processamento morfológico para segmentação cardíaca.
\end{abstract}


\section{Introdução}

A ecocardiografia bidimensional (eco 2D) é uma técnica de imagem ultrassônica amplamente utilizada no diagnóstico de doenças cardíacas. A detecção das bordas endocárdicas fornece uma maneira de medir áreas cardíacas, volumes e movimento, auxiliando no reconhecimento e avaliação de doenças cardiovasculares \cite{choy1996}.

Parâmetros como a fração de ejeção, derivados de medições de volume, são de grande valor clínico na avaliação do desempenho cardíaco. A confiabilidade da estimativa da fração de ejeção depende da precisão das medições de volume, que por sua vez dependem da precisão da detecção de bordas.

Este trabalho implementa o método proposto por Choy e Jin \cite{choy1996}, que combina filtragem morfológica com o operador de segunda derivada (Laplaciano da Gaussiana) para detecção de bordas endocárdicas do ventrículo esquerdo.

\section{Metodologia}

O algoritmo segue o pipeline ilustrado na Figura \ref{fig:pipeline}, composto por quatro etapas principais: (1) pré-processamento com filtragem morfológica, (2) detecção de bordas com filtro LoG, (3) segmentação watershed e (4) extração do contorno final.

\begin{figure}[H]
\centering
\includegraphics[width=0.4\textwidth]{figura_1.png}
\caption{Diagrama de blocos do algoritmo de extração de bordas (Choy \& Jin, 1996).}
\label{fig:pipeline}
\end{figure}

\subsection{Pré-processamento: Filtragem Morfológica}

A filtragem morfológica é utilizada para reduzir ruído e aumentar o contraste da imagem. O método considera o gráfico da imagem de eco 2D como uma superfície topográfica, onde o nível de intensidade de um pixel representa sua altitude.

Define-se a \textit{elevação} de um ponto $x$ como:
\begin{equation}
    elevacao(x) = altura(x) - altura(minimo(bacia(x)))
\end{equation}

onde $bacia(x)$ é a bacia que contém o ponto $x$, e $minimo(M)$ é o ponto de menor intensidade na bacia $M$.

Pontos com elevação menor que um limiar $h$ (tipicamente entre 15 e 55) são zerados, removendo o ruído na região da cavidade enquanto preserva as bordas.

\subsection{Detecção de Bordas: Filtro LoG}

O operador Laplaciano da Gaussiana (LoG) é aplicado para detecção de bordas:
\begin{equation}
    LoG(x,y) = -\frac{1}{\pi\sigma^4}\left[1 - \frac{x^2 + y^2}{2\sigma^2}\right]e^{-\frac{x^2+y^2}{2\sigma^2}}
\end{equation}

Os \textit{zero-crossings} (cruzamentos por zero) da imagem convoluída indicam as localizações das bordas com precisão sub-pixel.

\subsection{Segmentação Watershed}

A segmentação watershed é aplicada sobre os valores positivos da imagem convoluída. O algoritmo utiliza marcadores para:
\begin{itemize}
    \item Marcador 1: Região da cavidade do VE (a partir de um ponto semente)
    \item Marcador 2: Fundo/exterior da imagem
\end{itemize}

As linhas de watershed formam o contorno inicial, localizado a poucos pixels das bordas reais.

\subsection{Extração do Contorno}

O contorno final é extraído em três passos:
\begin{enumerate}
    \item Busca ao longo da direção normal por zero-crossings
    \item Busca na vizinhança para maximizar pontos do contorno
    \item Interpolação para preencher lacunas no contorno
\end{enumerate}

\section{Materiais e Métodos}

\subsection{Dataset}

Foi utilizado o dataset CAMUS (Cardiac Acquisitions for Multi-structure Ultrasound Segmentation) \cite{leclerc2019}, disponível gratuitamente sob licença CC BY-NC-SA 4.0 para uso em pesquisa científica não-comercial. O dataset contém imagens de ecocardiografia 2D com anotações de especialistas, incluindo:
\begin{itemize}
    \item Vistas apicais de 2 câmaras (2CH) e 4 câmaras (4CH)
    \item Frames de fim de diástole (ED) e fim de sístole (ES)
    \item Segmentações ground truth para cavidade do VE, miocárdio e átrio esquerdo
\end{itemize}

\subsection{Parâmetros}

Os parâmetros utilizados na implementação foram:
\begin{itemize}
    \item Limiar de elevação: $h = 25$
    \item Sigma do filtro LoG: $\sigma = 3.0$
    \item Raio de busca para zero-crossings: 5 pixels
\end{itemize}

\subsection{Métricas de Avaliação}

A avaliação seguiu a metodologia do paper original:

\textbf{Erro RMS} (Equação 2 do paper):
\begin{equation}
    RMS = \sqrt{\frac{1}{N}\sum_{i=1}^{N}d(L_{Ci}, L_{Hi})^2}
\end{equation}

onde $L_C$ são os pontos do contorno computado e $L_H$ os pontos do contorno manual (ground truth).

\textbf{Coeficiente Dice}:
\begin{equation}
    Dice = \frac{2|A \cap B|}{|A| + |B|}
\end{equation}

\textbf{Índice de Jaccard (IoU)}:
\begin{equation}
    IoU = \frac{|A \cap B|}{|A \cup B|}
\end{equation}

\section{Resultados}

\subsection{Resultados Quantitativos}

A Tabela \ref{tab:resultados} apresenta os resultados obtidos para cada imagem processada.

\begin{table}[H]
\centering
\caption{Resultados por imagem do dataset CAMUS}
\label{tab:resultados}
\begin{tabular}{lcccc}
\toprule
\textbf{Imagem} & \textbf{Erro RMS (px)} & \textbf{Dice} & \textbf{IoU} & \textbf{Razão Área} \\
\midrule
2CH\_ED & 14.07 & 0.849 & 0.738 & 1.043 \\
2CH\_ES & 15.46 & 0.704 & 0.543 & 1.176 \\
4CH\_ED & 18.24 & 0.869 & 0.769 & 0.897 \\
4CH\_ES & 15.19 & 0.769 & 0.624 & 0.881 \\
\midrule
\textbf{Média} & \textbf{15.74 $\pm$ 1.54} & \textbf{0.798 $\pm$ 0.066} & \textbf{0.669 $\pm$ 0.089} & \textbf{0.999 $\pm$ 0.116} \\
\bottomrule
\end{tabular}
\end{table}

\subsection{Comparação com o Paper Original}

A Tabela \ref{tab:comparacao} compara os resultados obtidos com os reportados no paper original.

\begin{table}[H]
\centering
\caption{Comparação com resultados do paper original}
\label{tab:comparacao}
\begin{tabular}{lcc}
\toprule
\textbf{Métrica} & \textbf{Este Trabalho} & \textbf{Choy \& Jin (1996)} \\
\midrule
Erro RMS & 15.74 px & 2.56 px \\
Desvio Padrão & 1.54 px & 1.21 px \\
Correlação de Área & $\sim$0.90 & 0.99 \\
\bottomrule
\end{tabular}
\end{table}

\textbf{Nota:} A diferença no erro RMS deve-se principalmente à diferença no tamanho das imagens. O paper original utilizou imagens de aproximadamente $128 \times 128$ pixels, enquanto as imagens do CAMUS possuem resolução de $549 \times 389$ pixels. Normalizando pelo tamanho, o erro relativo é comparável.

\subsection{Análise Visual}

A Figura \ref{fig:resultados} ilustra o pipeline completo de processamento para uma imagem 4CH em fim de diástole.

\begin{figure}[H]
\centering
\includegraphics[width=0.95\textwidth]{patient0001_4CH_ED_pipeline.png}
\caption{Resultados do processamento para imagem 4CH-ED: (a) Original, (b) Filtragem morfológica, (c) LoG, (d) Zero-crossings, (e) Watershed, (f) Contorno inicial, (g) Contorno final, (h) GT vs Computado.}
\label{fig:resultados}
\end{figure}

\section{Discussão}

O método implementado demonstrou resultados satisfatórios na segmentação do ventrículo esquerdo, com Dice médio de 0.798. A filtragem morfológica efetivamente remove ruído da região da cavidade e o filtro LoG detecta bordas com boa precisão. As principais limitações são: necessidade de um ponto semente dentro da cavidade (semi-automático), sensibilidade a dropouts nas imagens e parâmetros que podem necessitar ajuste para diferentes equipamentos.

\section{Conclusão}

A implementação do método de Choy e Jin (1996) para detecção de bordas endocárdicas demonstrou a viabilidade de técnicas clássicas de processamento morfológico para segmentação cardíaca. O coeficiente Dice de 0.798 indica boa concordância com as anotações de especialistas.

\begin{thebibliography}{9}

\bibitem{choy1996}
Choy, M. M., \& Jin, J. S. (1996). Morphological image analysis of left-ventricular endocardial borders in 2D echocardiograms. \textit{SPIE Vol. 2710}, 852-863.

\bibitem{leclerc2019}
Leclerc, S., Smistad, E., Pedrosa, J., Ostvik, A., et al. (2019). Deep Learning for Segmentation using an Open Large-Scale Dataset in 2D Echocardiography. \textit{IEEE Transactions on Medical Imaging}, 38(9), 2198-2210. doi: 10.1109/TMI.2019.2900516. Dataset disponível sob licença CC BY-NC-SA 4.0.

\end{thebibliography}

\end{document}